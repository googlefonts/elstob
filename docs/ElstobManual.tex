\documentclass[12pt,letterpaper,openany]{book}

\usepackage{ElstobManual}

\title{Elstob: A Variable Font for Medievalists}
\author{Peter S. Baker}

\begin{document}

\maketitle

\chapter{Introduction}

\pagestyle{fancy}
\fancyhead[CE]{\scshape\color{myRed} {\addfontfeatures{Numbers=OldStyle}\thepage}\hspace{10pt}%
\addfontfeature{Letters=UppercaseSmallCaps}\leftmark}
\fancyhead[CO]{\scshape\color{myRed} {elstob}\hspace{10pt}{\addfontfeatures{Numbers=OldStyle}\thepage}}

The Elstob font (named for the eighteenth-century medievalist Elizabeth Elstob)
is based on the Double Pica typeface commissioned by Bishop John Fell in the
seventeenth century and used by the Oxford University Press for many years.
Though it originates with an old-style typeface, it includes many modernizing touches,
and it exploits the latest in font technology so as to provide capabilities
available in few other fonts. It comes in two flavors: \textbf{static} and
\textbf{variable}.

\section{The Static Font}

Most of the fonts on your
computer are \textbf{static}: that is, each file contains outlines for a single style.
For example, the ubiquitous Times New Roman typically has four styles in four
files: regular, bold, italic, and bold italic. Because it has more than one
style, we call Times New Roman a \textbf{font family}. Think of this family as
having two stylistic \textbf{axes}, one being the \textbf{roman--italic} axis, on which
only two positions are available (since in serif typefaces there is no middle
ground between roman and italic), and
the other being the \textbf{Weight} axis, on which many positions are possible,
though the Times New Roman font family provides only two---regular and bold.
Some static font families come packaged in more than four files and provide both more
stylistic axes and more positions on these axes: these are called \textbf{extended
families}.

The static version of Elstob (ElstobD, where the D stands for “Desktop”) consists
of forty-eight files (don’t panic---you’re not required to install them all) and
has three stylistic
axes: \textbf{Roman--Italic}, \textbf{Weight}, and \textbf{Optical Size}. The
weights are {\extralight ExtraLight}, {\light Light}, Regular, {\medium Medium},
{\semibold SemiBold}, \textbf{Bold}, and {\extrabold ExtraBold}.

The \textbf{Optical Size} axis varies the shapes of characters to look good at
particular sizes. The static version comes in five optical sizes: 8pt, 10pt,
12pt (Regular), 14pt, and 18pt. You should choose the one that comes as close
as possible to the point size of the text you are setting---for example, “ElstobD
8pt” for footnotes and “ElstobD 14pt Medium” for subheads, depending on your
stylistic preferences.

\section{The Variable Font}

A \textbf{variable} font packages an almost infinite number
of styles into a single file. Whereas the static version of Elstob consists
of forty-eight files, the variable version consists of only two: Elstob.ttf
(roman) and Elstob-Italic.ttf (italic). And yet the variable version provides
many more stylistic choices than the static version because you can choose any
value along any of the font’s stylistic axes. The \textbf{weight} axis
runs from {\extralight 200 Extra Light} to {\extrabold 800 Extra Bold}. You
can choose either of those, or one of the other standard weights: {\light 300} for
{\light Light}, 400 for Regular, {\medium 500} for {\medium Medium},
{\semibold 600} for {\semibold SemiBold}, or \textbf{700} for \textbf{Bold}.
Or you can, if you’re so minded, choose {\oddball an oddball weight like
523.45}.

In addition to the \textbf{Weight} axis, Elstob has three others, \textbf{Optical
Size}, \textbf{Grade}, and \textbf{Spacing}, and the italic face has one more, \textbf{Slant}.
\textbf{Optical Size} works the same in the variable version as it does in the
static version, but you are not limited to five optical sizes, so you can tune
your type more closely to the size of your text. This axis runs from 6 to
18 (the numbers corresponding roughly to point sizes). Some applications (like
Adobe InDesign) will automatically
choose an appropriate value (though you may override it if you like).

\textbf{Grade}
(0--1) is like Weight, but it varies the weight of characters without changing
their widths. This can be useful in web pages, where dynamically changing the
Weight axis (for example, on mouseover) may cause text to reflow annoyingly. Grade will
not generally be useful in printed documents.

\textbf{Spacing} (0--1) increases the width of the space character and a few
related characters (such as non-breaking space). A value of zero (the default)
produces the original spacing of this font. A value of one produces the wider
spacing typical of documents printed in the nineteenth century and earlier.
Use this axis with \textSourceText{ss18} (“Old-style punctuation spacing”) to adhere closely to
the complex system of spacing typical of the era of metal type.

\textbf{Slant} (0--15) varies
the slant of the italic from {\italslanted 0 steeply slanted} to {\italupright
15 nearly upright}. The slant of the static fonts is
equivalent to \textit{a Slant of 6 in the variable font}.

The values of the various axes can be combined in any way you like. You will find,
however, that some combinations are problematic or inadvisable. For example,
you can add a little Grade to ExtraBold text to produce {\superbold even
bolder text}, but if you overdo it, letters will collide and become
misshapen---{\waytoobold like this}. You should add a Grade of no more than 0.2
to the ExtraBold face.
Combining ExtraBold and maximum Slant may also produce misshapen characters. It
is not unexpected that, in a three- or four-dimensional design space, some positions
will be less than optimal: experiment with the Elstob specimen page to find the
styles you like.

When beginning a project with Elstob, consider whether the software you’re using
has adequate support for variable fonts. If not, use the static version.
All of the major web browsers support variable fonts, which you can control via
CSS. The major programs of the Adobe Creative Suite also support variable fonts,
as does {\ltech}, used to produce this document;\footnote{Make sure to use the
latest version of {\ltech}, as support for variable fonts was added very
recently. XeTeX does not support variable fonts.} but most desktop applications
(such as word processors) do not. ElstobD, the static version of Elstob,
should work with all applications.

Elstob aims to cover all European languages using the Latin alphabet, and it includes
monotonic and polytonic Greek, both English and Scandinavian runes,
the basic International Phonetic Alphabet, and the most commonly used
mathematical symbols. It also has a curated collection of Unicode characters of
interest to medievalists. If you are interested in using Elstob in a project
but find that it lacks the characters you need, open an issue at the Elstob
development site.\footnote{https://github.com/psb1558/Elstob-font.}\pagebreak

\chapter{Specimens}

{\fontsize{72pt}{72pt}\selectfont ABCD\kern-2ptEFG\kern-2ptHI\\[0.3ex]
\fontsize{58}{58}\selectfont\textls[50]{JKLMNOPQRS}\\[0.1ex]
\fontsize{48}{48}\selectfont\textls[48]{TUVWXYZÞÐ\kern-5ptÆǷ}\\[0.4ex]
\fontsize{36}{36}\selectfont{}a\kern+0.5ptb\kern+0.5ptc\kern+0.5ptd\kern+0.5pte\kern+0.5ptfghijklmnopqrstuvwx\\[0.1ex]
Quousque tandem abutere, \textls[20]{Catilina, patientia nostra?}\linebreak
\italslanted \textls[20]{Qu\kern-1ptousque tandem abutere,\linebreak
Catilina, patientia nostra?}\linebreak
\textls[40]{0123456789}{\ }{\ } \ltab{\textls[40]{0123456789}}}


\section{Old English (Regular)}

{\large Sum wer wæs ġeseten on þām lande þe is ġehāten Hus; his nama wæs Iob. Se wer wæs swīðe bilewite and rihtwīs and ondrǣdende God and forbūgende yfel. Him wǣron ācennede seofan suna and ðrēo dohtra. Hē hæfde seofon ðūsend scēapa and ðrēo ðūsend olfenda, fīf hund ġetȳmu oxena and fīf hund assan and ormǣte miċelne hīred. Se wer wæs swīðe mǣre betwux eallum ēasternum, and his suna fērdon and ðēnode ǣlċ ōðrum mid his gōdum on ymbhwyrfte æt his hūse, and þǣrtō heora swustru ġelaðodon.}

\section{Old English with insular letter-forms (Light + ss02)}

{\large\light\addfontfeature{StylisticSet=2}Her on ðisum geare forðferde ælfgiue ymma Eadwardes cynges modor ⁊ hardacnutes cynges. ⁊ on þam sylfan geare gerædde se cyng ⁊ his witan ꝥ mann sceolde forðian ut to Sandwic scipu. ⁊ setton raulf eorl ⁊ oddan eorl to heafodmannū þærto. Ða gewende Godwine eorl út frā brycge mid his scypum to yseran. and let ut ane dæge ær midsumeres mæsseæfene ꝥ he cō to næsse. þe is besuðan rumenea. Þa cō hit to witenne þā eorlū ut to sandwic. ⁊ hi þa gewendon ut æfter þam oðrum scipum. ⁊ bead man landfyrde ut ongean þa scipu.}

\section{Old English in runes (various weights + ss12)}

{\extralight\addfontfeature{StylisticSet=12} fisc flodu ahof on fergenberig warþ gasric grorn þær he on greut giswom hronæs ban.
\light\addfontfeature{StylisticSet=12} fisc flodu ahof on fergenberig warþ gasric grorn þær he on greut giswom hronæs ban.
\medium\addfontfeature{StylisticSet=12} fisc flodu ahof on fergenberig warþ gasric grorn þær he on greut giswom hronæs ban.
\semibold\addfontfeature{StylisticSet=12} fisc flodu ahof on fergenberig warþ gasric grorn þær he on greut giswom hronæs ban.
\textbf{\addfontfeature{StylisticSet=12}fisc flodu ahof on fergenberig warþ gasric grorn þær he on greut giswom hronæs ban.}
\extrabold\addfontfeature{StylisticSet=12} fisc flodu ahof on fergenberig warþ gasric grorn þær he on greut giswom hronæs ban.}

\section{Middle English: \textit{Ancrene Wisse} (Medium)}

{\large\medium\addfontfeatures{StylisticSet=16,Language=English}\cvd[1]{38}{\cvd[1]{69}{Nan ancre bi mi read ne schal makien ꝓfessiun. ꝥ is bihaten ase heast⹎ but þreo þinges. ꝥ beoð obedience. chastete. ⁊ stude steaðeluestnesse. ꝥ ha ne schal ꝥ stude neau͛ mare changin bute for nede ane. as strengðe ⁊ deaðes dred. obedience of hire bischop oðer of his herre. for hƿa se nimeð þing on hond ⁊ bihat hit Godd as heast forte don hit⹎ ha bint hire þerto. ⁊ sunegeð deadliche i þe bruche. ȝef ha hit brekeð ƿilles. ȝef ha hit ne bihat naƿt. ha hit mei do þah ⁊ leauen hƿen ha ƿel ƿule.}}}
%as of mete. of drunch. flesch forgan oðer fisch. alle oþer sƿucche þinges. of ƿerunge. of liggunge. of ures. of oþre beoden.}}}

\section{Middle English: \textit{The Ormulum} (SemiBold + cv8[1], cv38[2], cv40[1], cv36[1])}

{\large\semibold\addfontfeatures{Language=English,CharacterVariant={8,38:1,40,36}}
Nu broþerr ƿallte͛ broþer\kern-6ptᫍ\kern+6pt{} mın. affte͛ þe flæshess kīde. ⁊
broþer\kern-6ptᫍ\kern+6pt{} mın ı crısstenndom.
þurrh fulluhht ⁊ þurrh \hlig{troƿƿþe}. ⁊ broþerr mın ı \cvd[1]{14}{goddess} hus.
\cvd{14}{ge̋t} o þe þrıde ƿıse.
Þurrh þatt ƿıt hafenn takenᷠ ba. an \cvd{14}{regͪell} boc to \cvd{14}{follgͪenn}.
Vnnderr kanunnkess had.
⁊ lıf. Sƿa sumͫ sannt aƿƿstın sette. Icͨcͨ hafe don sƿa sumͫ þu badd⹎ ⁊ forþedd te
þıᷠ ƿılle. ⹍ Icͨcͨ hafe ƿend ınᷠtıll \cvd[1]{14}{ennglıssh}. \cvd[1]{14}{goddspelless}
\cvd{14}{hallgͪe} láre⹎ Affte͛ ꝥ lıtᫎle
ƿıtt þatt me. mın drıhhtınn \hlig{hafeþþ} lenned.}

\section{Early Modern English (Light Italic)}

\textit{\large\itallight\addfontfeature{StylisticSet=8,Language=English}When the right vertuous E.W. and I were at the Emperours Court to\-gither, wee gave our selves to learne horsemanship of Jon Pietro Pugliano, one that with great commendation had the place of an Esquire in his stable: and hee according to the fertilnes of the Italian wit, did not onely affoord us the demonstration of his practise, but sought to enrich our \mbox{mindes} with the contemplations therein, which he thought most precious. But with none I remember mine eares were at any time more loaden, then when (either angred with slow paiment, or mooved with our learnerlike admiration) hee exercised his speech in the praise of his facultie.}

\section{Latin (Italic + cv38[2])}

\textit{\large\cvd[1]{38}{Humanas laudes et mortalium ınsulas uıdımus aut ére ıncıso conscrıptas· aut auro radıantıb: lıtterıs· ad posterıtatıs memorıam cōmendatas· Et ısta attendentes mıror quare non erubescımꝰ mılıtum xpı uıctorıas sılentıo tégere \& n̄ ad laudem ımperatorıs eoꝝ qualıt̄ pugnauerınt contra hostes \& uıcerınt· sedulıs saltım uılıbus tradere \& ad ıncıtandos anımos bellatoꝝ dılıgentıus explıcare· Multa bona talıū narratıonū scrıpta conuertant; Laus deı est cum ısta leguntur· memorıa scōꝝ excolıtur⹎ aedıfıcacıo m̄tıb: tradıtur. honor martırıbus exhıb\&ur·}}

\section{Old Icelandic (Medium Italic)}

{\italmedium\large Þá mælti Hárr: Þá er þeir gengu með sævarstrǫndu Borssynir, fundu þeir tré tvau ok tóku upp trén ok skǫpuðu af menn. Gaf inn fyrsti ǫnd ok líf, annarr vit ok hræring, þriði ásjónu, mál ok heyrn ok sjón, gáfu þeim klæði ok nǫfn. Hét karlmaðrinn Askr, en konan Embla, ok ólst þaðan af mannkindin, sú er byggðin var gefinn undir Miðgarði. Þar næst gerðu þeir sér borg í miðjum heimi, er kǫlluð er Ásgarðr. Þat kǫllum vér Trója.}

\section{Gothic (Medium Italic + Slant=0)}

{\italslantedmedium\large Warþ þan in dagans jainans, urrann gagrefts fram kaisara Agustau, gameljan allana midjungard. soh þan gilstrameleins frumista warþ at wisandin kindina Swriais raginondin Saurim Kwreinaiau. jah iddjedun allai, ei melidai weseina, ƕarjizuh in seinai baurg. Urrann þan jah Iosef us Galeilaia, us baurg Nazaraiþ, in Iudaian, in baurg Daweidis sei haitada Beþlaihaim, duþe ei was us garda fadreinais Daweidis, anameljan miþ Mariin sei in fragiftim was imma qeins, wisandein inkilþon.}

\section{Vietnamese (Regular + ss09)}

{\addfontfeature{Language=Vietnamese,StylisticSet=9}\large Ban đầu Ðức Chúa Trời dựng nên trời và đất. Thuở ấy đất hoang vắng và trống không. Bóng tối bao phử trên mặt vực thẳm. Thần[a] của Ðức Chúa Trời vận hành trên mặt nước. Ðức Chúa Trời phán, “Phải có ánh sáng,” thì có ánh sáng. Ðức Chúa Trời thấy ánh sáng là tốt đẹp. Ðức Chúa Trời phân rẽ giữa ánh sáng và bóng tối. Ðức Chúa Trời gọi ánh sáng là ngày và bóng tối là đêm. Vậy có hoàng hôn và bình minh – ngày thứ nhất.}

\section{French (Medium + Spacing=1, ss08, ss18)}

{\mediumspaced\addfontfeature{Language=French}\large Grandgousier était bon raillard en son temps,
aimant à boire net autant qu’homme qui pour lors fût
au monde, et mangeait volontiers salé. A cette fin,
avait ordinairement bonne munition de jambons de Mayence
et de Bayonne, force langues de bœuf fumées, abondance
d’andouilles en la saison et bœuf salé à la moutarde, renfort de
boutargues, provision de saucisses, non de Bologne, car il
craignait li boucon de Lombard, mais de Bigorre, de Longaunay,
de la Brenne et de Rouergue. En son âge virile, épousa
Gargamelle, fille du roi des Parpaillos, belle gouge» et de bonne
trogne, et faisaient eux deux souvent ensemble la bête à deux
dos, joyeusement se frottants leur lard, tant qu’elle engrossa
d’un beau fils, et le porta jusques à l’onzième mois.}

\section{Greek (Light)}

{\light\large\addfontfeatures{Script=Greek,Language=Greek} Ἐγένετο δὲ ἐν ταῖς ἡμέραις ἐκείναις ἐξῆλθεν δόγμα παρὰ Καίσαρος 
Αὐγούσ\-του ἀπογράφεσθαι πᾶσαν τὴν οἰκουμένην. αὕτη ἀπογραφὴ πρώτη 
ἐγένετο ἡγε\-μονεύοντος τῆς Συρίας Κυρηνίου. καὶ ἐπορεύοντο πάντες 
ἀπογράφεσθαι, ἕκαστος εἰς τὴν ἑαυτοῦ πόλιν. Ἀνέβη δὲ καὶ Ἰωσὴφ ἀπὸ 
τῆς Γαλιλαίας ἐκ πόλεως Ναζαρὲθ εἰς τὴν Ἰουδαίαν εἰς πόλιν Δαυὶδ 
ἥτις καλεῖται Βηθλέεμ, διὰ τὸ εἶναι αὐτὸν ἐξ οἴκου καὶ πατριᾶς Δαυίδ,
ἀπογράψασθαι σὺν Μαριὰμ τῇ ἐμνηστευμένῃ αὐτῷ, οὔσῃ ἐγκύῳ. ἐγένετο δὲ 
ἐν τῷ εἶναι αὐτοὺς ἐκεῖ ἐπλήσθησαν αἱ ἡμέραι τοῦ τεκεῖν αὐτήν, καὶ 
ἔτεκεν τὸν υἱὸν αὐτῆς τὸν πρωτότοκον· καὶ ἐσπαργάνωσεν αὐτὸν καὶ 
ἀνέκλινεν αὐτὸν ἐν φάτνῃ, διότι οὐκ ἦν αὐτοῖς τόπος ἐν τῷ καταλύματι.}

\section{Greek {Bold Italic}}

\textit{\textbf{\large Ἐγένετο δὲ ἐν ταῖς ἡμέραις ἐκείναις ἐξῆλθεν δόγμα παρὰ Καίσαρος 
Αὐγούστου ἀπογράφεσθαι πᾶσαν τὴν οἰκουμένην. αὕτη ἀπογραφὴ πρώτη 
ἐγένετο ἡγεμονεύοντος τῆς Συρίας Κυρηνίου. καὶ ἐπορεύοντο πάντες 
ἀπογράφεσθαι, ἕκαστος εἰς τὴν ἑαυτοῦ πόλιν. Ἀνέβη δὲ καὶ Ἰωσὴφ ἀπὸ 
τῆς Γαλιλαίας ἐκ πόλεως Ναζαρὲθ εἰς τὴν Ἰουδαίαν εἰς πόλιν Δαυὶδ 
ἥτις καλεῖται Βηθλέεμ, διὰ τὸ εἶναι αὐτὸν ἐξ οἴκου καὶ πατριᾶς Δαυίδ,
ἀπογράψασθαι σὺν Μαριὰμ τῇ ἐμνηστευμένῃ αὐτῷ, οὔσῃ ἐγκύῳ. ἐγένετο δὲ 
ἐν τῷ εἶναι αὐτοὺς ἐκεῖ ἐπλήσθησαν αἱ ἡμέραι τοῦ τεκεῖν αὐτήν, καὶ 
ἔτεκεν τὸν υἱὸν αὐτῆς τὸν πρωτότοκον· καὶ ἐσπαργάνωσεν αὐτὸν καὶ 
ἀνέκλινεν αὐτὸν ἐν φάτνῃ, διότι οὐκ ἦν αὐτοῖς τόπος ἐν τῷ καταλύματι.}}



\chapter{OpenType features}

OpenType is the format employed by most modern fonts. It enables such technical
wizardry as ligatures, kerning, and several kinds of variation. OpenType features,
when they can be controlled by users, can be selected via four-character tags.
Some applications offer more access to these features than others. The major web browsers
support all of Elstob’s features, and so do LibreOffice, Affinity Publisher,
XeTeX and {\ltech}. The Adobe Creative Suite supports a generous selection of
them. Microsoft Word, unfortunately, supports only a few OpenType features.

Elstob’s OpenType
features are for the most part a subset of those of Junicode. It will be noted
below when the two fonts differ. Features are presented in alphabetical order,
but this is not the order in which they are executed when more than one feature
has been applied.

Different applications provide different ways of accessing OpenType features.
Those that are available in Microsoft Word can be accessed in the “Advanced”
tab of the “Font” dialog. In the Adobe apps those features that are available
can be accessed via the “Character” dialog, and in InDesign via the “O” icon that
appears when text is selected (InDesign users should select the “World-Ready
Paragraph Composer” when using this font). For variable font handling in
{\ltech}, see the source for this document.

\section{aalt (Access All Alternates)}
Provides access to all variants in the font. Applications that use this feature
usually do so via an element of the user interface.

\section{c2sc (Small Capitals From Capitals)}
Converts capitals to small caps. Every capital in the font has a corresponding
small capital. ABCDÞÐÆ → {\addfontfeature{Letters=UppercaseSmallCaps} ABCDÞÐÆ}.

\section{calt (Contextual Alternates)}
In most applications this feature is on by default.
Provides many alternate characters that vary automatically by context.

\section{case (Case-Sensitive Forms)}
Mostly provides alternate diacritics for capitals, e.g. ÂÄÉËÒÕŰŪ. Also converts
old-style to lining figures to harmonize with capitals. Some applications turn
this feature on automatically in the vicinity of capitals.

\section{ccmp (Glyph Composition/Decomposition)}
In most applications this feature is on by default and cannot be turned off.
In Elstob it performs (1.) removal of dot from i and j when followed by combining
marks; (2.) substituting ligatures for certain vowel + rhotic hook (\unic{U+02DE})
combinations; and (3.) substituting precomposed characters for letter +
mark sequences in polytonic Greek.

\section{cv07 (Variants of D)}
Provides insular D (\cvd{7}{D}).

\section{cv08 (Variants of d)}
Provides two shapes of insular d: 1. \cvd{8}{d}; 2. \cvd[1]{8}{d}.

\section{cv11 (Variants of F)}
Provides insular F (\cvd{11}{F}).

\section{cv12 (Variants of f)}
Provides: 1. \cvd{12}{f} (insular f); 2. \cvd[1]{12}{f} (narrow f).

\section{cv13 (Variants of G)}
Provides: 1. \cvd{13}{G} (insular G); 2. \cvd[1]{13}{G} (Orm’s hard G).

\section{cv14 (Variants of g)}
Provides: 1. \cvd{14}{g} (insular g); 2. \cvd[1]{14}{g} (Orm’s hard g);
3. \cvd[2]{14}{g} (script g)

\section{cv18 (Variants of i)}
Provides dotless i (\cvd{18}{i}).

\section{cv35 (Variants of R)}
Provides insular R (\cvd{35}{R}).

\section{cv36 (Variants of r)}
Provides: 1. \cvd{36}{r} (insular r); 2. \cvd[1]{36}{r} (r rotunda).

\section{cv37 (Variants of S)}
Provides insular S (\cvd{37}{S}).

\section{cv38 (Variants of s)}
Provides: 1. \cvd{38}{s} (insular s); 2. \cvd[1]{38}{s} (long s);
3. \cvd[2]{38}{s} (narrow long s). Instances of \textbf{\cvd[1]{38}{s}}
provided by this feature are not subject to the contextual rules followed
when \textSourceText{ss08} is turned on. Use \textSourceText{cv38[1]} for \textbf{\cvd[1]{38}{s}}
everywhere in the text, or enter \unic{\unic{U+017F}} directly for fine control over the
distribution of \textbf{\cvd[1]{38}{s}}. Use \textSourceText{cv38[2]} to avoid collisions that
Elstob’s contextual rules have not anticipated.

\section{cv39 (Variants of T)}
Provides insular T (\cvd{39}{T}).

\section{cv40 (Variants of t)}
Provides insular t (\cvd{40}{t}).

\section{cv57 (Variants of æ)}
Italic face only. Provides an alternative (and in some contexts less ambiguous)
\textit{æ} (\textit{\cvd{57}{æ}}).
This feature also affects \unic{\unic{U+01E3}} (\textit{\cvd{57}{ǣ}})
and \unic{\unic{U+01FD}} (\textit{\cvd{57}{ǽ}}).

\section{cv69 (Variants of Tironian et sign ⁊)}
Provides two variants of \unic{\unic{U+204A}}: 1. \cvd{69}{⁊}; 2. \cvd[1]{69}{⁊}.

\section{cv76 (Variants of ?)}
Provides the \textit{punctus interrogativus} (\cvd{76}{?}).

\section{cv80 (variants of bracket characters)}
In the italic face, provides upright variants of the most common bracket
characters: (\ ) [\ ] \{\ \} ⟨\ ⟩.

\section{dlig (Discretionary Ligatures)}
Provides {\addfontfeature{Ligatures=Rare} ct} and
{\addfontfeature{Ligatures=Rare} st} ligatures, and in italic only,
{\addfontfeature{Ligatures=Rare} \textit{as}, \textit{is}, \textit{us}}.

\section{frac (Fractions)}
Elstob includes only three fractions: {\addfontfeature{Fractions=On} 1/4, 1/2,
3/4}. Type as number + slash + number.

\section{hlig (Historical Ligatures)}
Provides several ligatures used in Orm’s orthography:
{\addfontfeatures{Ligatures=Historic} þþ ƿƿ hh pp} for
{\addfontfeatures{Language=English}þþ ƿƿ hh pp}. The first two of these correspond
to Unicode \unic{\unic{U+A7D3}} and \unic{\unic{U+A7D5}}, but using the ligatures provides an intelligible
fallback when a font with these rare Unicode characters is not available.

\section{liga (Standard Ligatures)}
Most of this font’s ligatures are Contextual Alternates (\textSourceText{calt}), but a few
are provided by this feature, which should always be on.

\section{locl (Localized Forms)}
In most applications this feature is on by default and cannot be turned off.
It provides the English forms of thorn and eth
(\textbf{\addfontfeature{Language=English}{Þ þ ð}}) when English is the
active language. It also provides variants for several languages supported by this font:
Azerbaijani, Catalan, Greek, Kazakh, Romanian, Turkish, Vietnamese, and a few minor
languages.

\section{ordn (Ordinals)}
Provides superscript forms of \textbf{a} and \textbf{o} when preceded by a figure:
{\addfontfeature{VerticalPosition = Ordinal} 1a, 2o}.

\section{smcp (Small Capitals)}
Converts lowercase letters to small capitals. abcdeþðæ → \textsc{abcdeþðæ}.

\section{ss01 (Alternate Thorn and Eth)}
Overrides any language setting to provide alternate shapes of the letters thorn and eth: Nordic shapes when the language is English, and English shapes otherwise.

\section{ss02 (Insular Letter-Shapes)}
Transliterates from modern to insular (Old English, Old Irish) letter-shapes:\linebreak
dfgirstw → {\addfontfeature{StylisticSet=2} dfgirstw}. Note that \textSourceText{calt} later
changes the sequence \textbf{\addfontfeature{StylisticSet=2} s\kern0ptt}
to \textbf{\addfontfeature{StylisticSet=2} st}. You can override this behavior by placing
\unic{\unic{U+200C}} \textsc{zero-width non-joiner} or any invisible formatting mark
between the \textbf{s} and the \textbf{t}.

\section{ss03 (Alternate Figures)}
Changes oldstyle one and zero to more modern forms: a peaked one and a weighted zero. This
feature also affects superscripts, subscripts, and the slashed zero.

\section{ss04 (IPA Letter-Shapes)}
Changes \textbf{g} to \textbf{ɡ} and (in italic only)
\textbf{\textit{a}} to \textbf{\textit{\addfontfeature{StylisticSet=4} a}}. Some Greek letters
are changed to shapes that harmonize with IPA characters.

\section{ss08 (Contextual Long s)}
In English, French, Italian, and Spanish text, and in combination with \textSourceText{calt}, distributes \textbf{s} and
\textbf{\cvd[1]{38}{s}}
according to rules commonly employed by early printers in each language. For all other languages,
\textbf{s} and \textbf{\cvd[1]{38}{s}} are distributed according to the following rules:
\textbf{s} in word-final position and immediately before or after \textbf{f};
\textbf{\cvd[1]{38}{s}} everywhere else. To suppress any instance of \textbf{\cvd[1]{38}{s}},
place \unic{U+200C} \textsc{zero-width non-joiner} immediately after.

\section{ss09  (Language-specific variants)}
This feature is reserved for stylistic variants that occur in particular language
systems. At present only two languages are supported here: in English the feature
selects an alternative form of \textbf{insular d}, and in Vietnamese it selects forms of
accented \textbf{i} that retain the dot.

\section{ss12 (Early English Futhorc)}
Transliterates Latin script to runic with characters from the Early English futhorc. fuþorc
→ {\addfontfeature{StylisticSet=12} fuþorc}.

\section{ss13 (Elder Futhark)}
Transliterates Latin script to runic with characters from the Elder futhark. fuþark
→ {\addfontfeature{StylisticSet=13} fuþark}.

\section{ss14 (Younger Futhark)}
Transliterates Latin script to runic with characters from the Younger futhark. fuþark
→ {\addfontfeature{StylisticSet=14} fuþark}.

\section{ss15 (Long Branch to Short Twig)}
Use with \textSourceText{ss14}. Converts the default (Long Branch) version of the Younger futhark
to the Short Twig version. fuþark
→ {\addfontfeature{StylisticSet=14,StylisticSet=15} fuþark}.

\section{ss16 (Contextual r Rotunda)}
Together with \textSourceText{calt}, distributes \textbf{r} and
\textbf{\cvd[1]{36}{r}} in accordance with the rules most often employed in medieval manuscripts and early printed books: {\addfontfeature{StylisticSet=16} form workrooms priest prayer}.

\section{ss18 (Old-style punctuation spacing)}
{\spaced\addfontfeature{StylisticSet=18} Adds extra space inside paired quotation marks and before semicolons, colons, question
marks and exclamation marks. The width of a space between a sentence-ending sequence
(e.g. period, period + quotation mark, question mark) and a capital letter is increased.
The amount of space added in these environments is governed by the Spacing (SPAC) axis,
which runs from 0 (the default) to 1. When Spacing is set to 1, the spacing between
words and sentences and around punctuation marks is a good match for most books
printed in the late eighteenth century.

This feature will produce too much space in certain sequences that can be mistaken
for the end of a sentence (like “Ofc. Smith”—though a number
honorifics in various languages are accounted for in the rules. To solve this problem, place a}
\textsc{zero-width non-joiner }{\spaced\addfontfeature{StylisticSet=18}(\unic{\unic{U+200C}}) anywhere between the period and the capital
or replace the space with the non-breaking space (\unic{U+00A0}) or thin space (\unic{U+2009}) (“Ofc.^^a0Smith,”^^^^2009“Ofc.^^^^2009Smith”). If
the rules produce a narrow space between sentences where you want a wide one,
place a }\textsc{zero-width non-joiner} {\spaced\addfontfeature{StylisticSet=18}before the period (“Main St\/. And”).
You can also override the effect of this feature by using the font’s alternative
spaces: em space (\unic{U+2003}), en space (\unic{U+2002}), hair space (\unic{U+200A}), thin space (\unic{U+2009}),
or three-per-em space (\unic{U+2004}). Find rules for using these spaces in
handbooks for compositors from the era of metal type.

\textSourceText{ss18} and the Spacing axis will fail in a few OpenType-aware applications (including Safari) that
handle spaces in a non-standard way. In these applications some spaces that should be increased
will remain unchanged.}


\section{subs (Subscripts)}
Subscript numbers, both lining and old style. 01234 →
{\addfontfeature{VerticalPosition=Inferior} 01234};
\ltab{01234 →
{\addfontfeature{VerticalPosition=Inferior} 01234}}.

\section{sups (Superscripts)}
Superscript numbers, both lining and old style. 56789 →
{\addfontfeature{VerticalPosition=Superior} 56789};
\ltab{56789 →
{\addfontfeature{VerticalPosition=Superior} 56789}}.

\section{swsh (Swash)}
Italic only. Provides swash forms of certain capitals
{\addfontfeature{Style=Swash}(\textit{A D J P R T} ) plus \textit{z} and \textit{k}.}

\section{tnum (Tabular Figures), onum (Old-Style Figures), pnum (Proportional
Figures), lnum (Lining Figures)}
In various combinations, provides figures in four styles:
tabular lining (the default),
tabular old-style,
proportional lining,
proportional old-style. The font contains variants of its mathematical
operators to harmonize with the old-style figures.
% For some reason, proportional lining causes an error in luaotfload.
% So we do without examples until it can be worked out.

\section{zero (Slashed Zero)}
Provides slashed zero ({\addfontfeature{Numbers=SlashedZero} 0 \ltab{0}}) in all figure styles.

\chapter{Greek}

As of version 3.0, Elstob supports both modern and ancient Greek.
When typing accented characters in modern Greek, be sure to use
forms with tonos (Ά \unic{U+0386}, ά \unic{U+03AC}, Έ \unic{U+0388}, έ \unic{U+03AD}, Ή \unic{U+0389}, ή \unic{U+03AE},
Ί \unic{U+038A} ί \unic{U+03AF}, ΐ \unic{U+0390}, Ό \unic{U+038C}, ό \unic{U+03CC}, Ύ \unic{U+038E}, ύ \unic{U+03CD}, ΰ \unic{U+03B0},
Ώ \unic{U+038F}, ώ \unic{U+03CE}). When typing polytonic Greek, instead
use forms with oxia (Ά \unic{U+1FBB}, ά \unic{U+1F71}, Έ \unic{U+1FC9}, έ \unic{U+1F73}, Ή \unic{U+1FCB}, ή \unic{U+1F75},
Ί \unic{U+1FDB}, ί \unic{U+1F77}, ΐ \unic{U+1FD3}, Ό \unic{U+1FF9}, ό \unic{U+1F79},
Ύ \unic{U+1FEB}, ύ \unic{U+1F7B}, ΰ \unic{U+1FE3},
Ώ \unic{U+1FFB}, ώ \unic{U+1F7D}). Although the tonos and oxia forms look the same, they
behave differently: in a sequence of capital letters, or if you capitalize lowercase
letters, the tonos disappears,
in accordance with the modern rule; but the same does not happen with the oxia forms.

Elstob has a full set of precomposed polytonic forms, but if you like you can
type a base letter followed by the appropriate diacritics. For example, the sequence
α \unic{U+03B1} ◌ͅ \unic{U+0345} ◌᾿ \unic{U+1FBF} ◌` \unic{U+1FEF} will yield 
{\addfontfeature{Script=Greek,Language=Greek}ᾳ᾿`}, 
and the sequence Α \unic{U+0391}
ι \unic{U+1FBE} ◌῾ \unic{U+1FFE} ◌´ \unic{U+1FFD} will yield
{\addfontfeature{Script=Greek,Language=Greek}Αι῾´}.
Characters must come in this order:
base character, prosgegrammeni \unic{U+1FBE} or ypogegrammeni \unic{U+0345}, and then other
diacritics from left to right or bottom to top.
If it's easier, you can type ◌́ \unic{U+0301} for oxia, ◌̀ \unic{U+0300} for varia,
◌̕ \unic{U+0315} for psili, ◌̔ \unic{U+0314}
for dasia, and ◌̆ \unic{U+0306} for vrachy. Elstob's OpenType programming
will substitute the correct characters.\footnote{Always make sure the language for
your document (or for a Greek passage) is set correctly. With fontspec, both
Script and Language must be set to Greek.}

Some users may be tempted to substitute Latin capitals for Greek capitals that
look the same (e.g. B, H, T). Doing so, especially in polytonic text, will
cause problems that may be difficult to diagnose.


\chapter{Unicode character list}

This is a list of characters with Unicode encodings in Elstob. In addition, the
font contains more than 600 unencoded characters, including small
caps, ligatures, and symbols, accessible via OpenType features (listed above).

Code points for which Elstob has no glyphs are represented in the table by blue
bullets (the actual bullet at U+2022 is black).
Many of Elstob's glyphs (e.g. spaces, formatting marks) are invisible: these
are represented by blanks in the table. A few glyphs are too large for their table cells,
and these spill out on one or more sides.

\displayfonttable[color=blue,title-format=\caption{Encoded Glyphs in Elstob},
title-format-cont=\caption{Encoded Glyphs in Elstob, \emph{cont.}}, missing-glyph=•,
missing-glyph-color=blue, glyph-width=12pt, hex-digits=head]{Elstob.ttf}[Renderer=HarfBuzz]

\end{document}
